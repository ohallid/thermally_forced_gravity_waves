\documentclass[12pt]{article}

\usepackage{cite}
\usepackage{amstext}
\usepackage{amssymb}
\usepackage{amsmath}
%\usepackage{amssymb}
\usepackage{epsfig}
\usepackage{graphics}
\usepackage{graphicx}

\DeclareMathOperator\erf {erf}


\bibliographystyle{plain}

\begin{document}

\begin{titlepage}

\vfill \LARGE
\begin{center}
Potential Tempertature Investigations, Parker and Burton System
\large

\rule{0mm}{20mm}

O. J.  Halliday, S. D. Griffith,  D. Parker,

 \vspace{3mm} {\em University of Leeds}


%\rule{40mm}{0.2mm}



\end{center}

\pagestyle{empty}
\end{titlepage}
%
%
%
\section{Preliminary Results and Error Function Properties}
\label{sec_lemma}
%
We shall need to evaluate integrals $I$ with the following form:
%
\begin{equation}
I (t,T,c)= \int_0^t H(t'-T) \exp \left(- \frac{(\xi - c t')^2 }{ 2 \sigma^2} \right) dt', \quad \forall t>0.
\end{equation}
%
Using the defintion of the Heavyside function $I$ may be written :
%
\begin{equation}
I (t,T,c)= H(t-T) \int_T^t \exp \left(- \frac{(\xi - c t')^2 }{ 2 \sigma^2} \right) dt', \quad \forall t>0,
\end{equation}
%
which, on using the substitution $u = \frac{ct'-\xi}{\sqrt{2} \sigma}$ transforms to :
%
\begin{equation}
I (t,T,c) = H(t-T) \frac{\sqrt{2} \sigma }{ c } \int_{ \frac{cT-\xi}{\sqrt{2} \sigma}}^{ \frac{ct-\xi}{\sqrt{2} \sigma} } \exp \left(- u^2 \right) du, \quad \forall t>0.
\end{equation}
%
In order to appeal to the definition of the error function we must transform the right hand side in the above:
%
\begin{equation}
I (t,T,c) = H(t-T) \frac{\sqrt{2} \sigma }{ c }\left(  \int_{ \frac{cT-\xi}{\sqrt{2} \sigma}}^0 \exp \left(- u^2 \right) du +  \int_0^{ \frac{ct-\xi}{\sqrt{2} \sigma} } \exp \left(- u^2 \right) du \right) .
\end{equation}
%
Reversing limits in the first term on teh right hand side:
%
\begin{equation}
I (t,T,c) = H(t-T) \frac{\sqrt{2} \sigma }{ c }\left(  - \int_0^{ \frac{cT-\xi}{\sqrt{2} \sigma}} \exp \left(- u^2 \right) du +  \int_0^{ \frac{ct-\xi}{\sqrt{2} \sigma} } \exp \left(- u^2 \right) du \right) .
\end{equation}
%
Employing the defintion of the error function we may obtain an expression for the integrals in the above:
%
\begin{equation}
\erf (x) \equiv  \frac{ 2 }{\sqrt{ \pi } } \int_0^x \exp \left(- u^2 \right) du \iff  \int_0^x \exp \left(- u^2 \right) du = \frac{\sqrt{\pi}}{2} \erf(x),
\end{equation}
%
we easily obtain the following:
%
\begin{equation}
I (t,T,c) = H(t-T) \sqrt{ \frac{\pi}{2} } \frac{ \sigma }{ c }\left( - \erf \left( \frac{cT-\xi}{\sqrt{2} \sigma} \right) +  \erf \left( \frac{ct-\xi}{\sqrt{2} \sigma} \right)    \right) .
\end{equation}
%
We note some results which will be useful in what follows. 

First, the error function is an odd function of its argument:
%
\begin{equation}
\erf(-x) = - \erf(x).
\end{equation}

Second, set $T=0$ and use that fact to obtain:
%
\begin{equation}
I (t,0,c) = \sqrt{ \frac{\pi}{2} } \frac{ \sigma }{ c } \left(  \erf \left( \frac{\xi}{\sqrt{2} \sigma} \right) +  \erf \left( \frac{ct-\xi}{\sqrt{2} \sigma} \right)    \right) .
\end{equation}
%

Third, change the sign of $c$ (i.e. $c \rightarrow (-c) $ ) to obtain:
%
\begin{eqnarray}
\label{equ_property1}
I (t,0,-c) & = & \sqrt{ \frac{\pi}{2} } \frac{ \sigma }{(- c) }\left(  \erf \left( \frac{\xi}{\sqrt{2} \sigma} \right) + \erf \left( \frac{- ct-\xi}{\sqrt{2} \sigma} \right)    \right)  \\ \nonumber
& = &  \sqrt{ \frac{\pi}{2} } \frac{ \sigma }{ c }\left( -  \erf \left( \frac{\xi}{\sqrt{2} \sigma} \right) + \erf \left( \frac{ct+\xi}{\sqrt{2} \sigma} \right)    \right)
\end{eqnarray}
%
\section{Potential Temperature for a System with Transient Heating}
\label{sec_bresponse}
%
%
%
We note that a steady response is a special case of a transient response. We therefore exploit previous experience and consider the unsteady directly.
We shall assume a time variation of heating as:
%
\begin{equation}
 H(t) - H(t-T) 
\end{equation}
%
Throughout we shall consider the Parker and Burton system and assume $s=0$ (heat source is stationary) and $n=f=0$.
Also, we shall assume that the horizontal variation of heating is given by:
%
\begin{equation}
F(x) \equiv \exp \left( - \frac{x^2}{2 \sigma^2 } \right).
\end{equation}
%
%
%
\subsection{ Previous Work, Steady Heating $w$-Response}
%
The steady response, $w_{m_z}$, to the $m_z$ mode of a steady (constant) heating function in a Parker and Burton systemmay be written:
%
\begin{equation}
\frac{w_{m_z} }{Q} = \frac{H^2}{m_z^2 \pi^2} \left( \frac{F(x)}{c^2} - \frac{1}{2} \frac{F(x-ct)}{c^2 } - \frac{1}{2} \frac{F(x+ct)}{ c^2}   \right) \sin \left( \frac{m_z \pi} {H} z\right),
\end{equation}
%
where:
%
\begin{equation}
 c = \frac{N H}{ m_z \pi }, \quad m_z \in \mathbb{Z}.
\end{equation}
%
The above may be simplified:
%
\begin{equation}
\frac{w_{m_z} }{Q} = \frac{1}{N^2} \left( F(x) - \frac{1}{2} F(x-ct)  - \frac{1}{2} F(x+ct)   \right) \sin \left( \frac{m_z \pi} {H} z\right),
\end{equation}
%
For a steady heating function:
%
\begin{equation}
S(x,z,t) =Q H(t) F(x) \sum_{m_z=1}^{\infty} b_{m_z} \sin \left( \frac{m_z \pi} {H} z\right),
\end{equation}
%
the aggregate $w$-response is obtained by superposing modes:
%
\begin{equation}
\frac{w }{Q} = \frac{1}{N^2}  \sum_{m_z = 1}^{\infty} b_{m_z} \left( F(x) - \frac{1}{2} F(x-ct)  - \frac{1}{2} F(x+ct)   \right) \sin \left( \frac{m_z \pi} {H} z\right),
\end{equation}
%
which can be written, note:
%
\begin{equation}
\frac{w }{Q} =  \frac{S_0}{Q N^2} - \frac{1}{2 N^2}  \sum_{m_z = 1}^{\infty} b_{m_z} \left(F(x-ct) + F(x+ct)   \right) \sin \left( \frac{m_z \pi} {H} z\right),
\end{equation}
%
where we have defined:
%
\begin{equation}
S_0 (x,z)  \equiv Q F(x) \sum_{m_z=1}^{\infty} b_{m_z} \sin \left( \frac{m_z \pi} {H} z\right),
\end{equation}
%
\subsection{ Previous Work, Unsteady Heating $w$-Response}
%
%
%
Let us now consider transient heating, still with $s=0$, $n = f = 0$. Using a Fourier series expansion to express the vertical variation, a transient heating function of duration $T$ is:
%
\begin{equation}
\label{equ_1}
S(x,z,t) =Q (H(t) - H(t-T) ) F(x) \sum_{m_z=1}^{\infty} b_n \sin \left( \frac{m_z \pi} {H} z\right),
\end{equation}

Using the delay theorem of Laplace transforms, the corresponding $w$-response may be computed:
%
\begin{eqnarray}
\label{equ_2}
\frac{w }{Q} & = & \frac{S_0}{Q N^2} - \frac{1}{2 N^2}  \sum_{m_z = 1}^{\infty} b_{m_z} \left(F(x-ct)  + F(x+ct)   \right) \sin \left( \frac{m_z \pi} {H} z\right) \\ \nonumber
& - & \frac{S_0}{Q N^2}  H(t-T) \\ \nonumber
& + & \frac{1}{2 N^2} H(t-T) \sum_{m_z = 1}^{\infty} b_{m_z} \left(F(x-c(t-T)) + F(x+c(t-T))   \right) \sin \left( \frac{m_z \pi} {H} z\right) \\ \nonumber
\end{eqnarray}
%
%
%
\subsection{ Unsteady Heating $b$-Response}
%
Using equation (4) of Parker and Burton, the $b$-response may be written:
%
\begin{equation}
\frac{\partial  }{ \partial t } \left(  \frac{b}{Q} \right) =  \frac{S}{Q} - N^2 \frac{w}{Q},
\end{equation}
%
which, on using equations \ref{equ_1} and \ref{equ_2} gives:
%
\begin{eqnarray}
\frac{\partial  }{\partial t} \left( \frac{b}{Q} \right) & = &  F(x) \sum_{m_z=1}^{\infty} b_n \sin \left( \frac{m_z \pi} {H} z\right) \\ \nonumber
& - & H(t-T) F(x) \sum_{m_z=1}^{\infty} b_n \sin \left( \frac{m_z \pi} {H} z\right) \\ \nonumber
& - & \frac{S_0}{Q} \\ \nonumber 
& +& \frac{1}{2}  \sum_{m_z = 1}^{\infty} b_{m_z} \left(F(x-ct)  + F(x+ct)   \right) \sin \left( \frac{m_z \pi} {H} z\right) \\ \nonumber
& + &H(t-T)  \frac{S_0}{Q}  \\ \nonumber
& - & \frac{1}{2} H(t-T) \sum_{m_z = 1}^{\infty} b_{m_z} \left(F(x-c(t-T)) + F(x+c(t-T))   \right) \sin \left( \frac{m_z \pi} {H} z\right), \\ \nonumber
\end{eqnarray}
%
where we have suppressed a factor $H(t)$ in the first term, note. 
Consider the right hand side of the above. Using the definition of $S_0$ above the first and third cancel, as do the second and fifth:
%
\begin{eqnarray}
\frac{\partial  }{\partial t} \left( \frac{b}{Q} \right) & = & \frac{1}{2} H(t) \sum_{m_z = 1}^{\infty} b_{m_z} \left(F(x-ct) + F(x+ct)   \right) \sin \left( \frac{m_z \pi} {H} z\right) \\ \nonumber
& - & \frac{1}{2} H(t-T) \sum_{m_z = 1}^{\infty} b_{m_z} \left(F(\xi-ct)  + F( \xi'+ct)   \right) \sin \left( \frac{m_z \pi} {H} z\right),  \\ \nonumber
\end{eqnarray}
%
where we have defined: 
%
\begin{equation}
\xi = x + cT, \quad \xi' = x - cT,
\end{equation}
%
and, recall $F(x) \equiv \exp \left( - \frac{x^2}{2 \sigma^2 } \right)$.  Integrating on $t$ now:

%
\begin{eqnarray}
\frac{b}{Q}  & = & \frac{1}{2} \sum_{m_z = 1}^{\infty} b_{m_z}\sin \left( \frac{m_z \pi} {H} z\right)\int_0^t H(t') \exp \left( - \frac{ (x-ct')^2 }{2 \sigma^2 } \right) dt'  \\ \nonumber
& + & \frac{1}{2} \sum_{m_z = 1}^{\infty} b_{m_z}  \sin \left( \frac{m_z \pi} {H} z\right) \int_0^t H(t') \exp \left( - \frac{ (x+ct')^2 }{2 \sigma^2} \right) dt' \\ \nonumber
& - & \frac{1}{2} \sum_{m_z = 1}^{\infty} b_{m_z} \sin \left( \frac{m_z \pi} {H} z\right) \int_0^t H(t'-T) \exp \left( - \frac{ (\xi-ct')^2 }{2 \sigma^2} \right) dt'  \\ \nonumber
& - & \frac{1}{2} \sum_{m_z = 1}^{\infty} b_{m_z}  \sin \left( \frac{m_z \pi} {H} z\right) \int_0^t  H(t'-T) \exp \left( - \frac{ ( \xi'+ct')^2 }{2 \sigma ^2} \right) dt', \\ \nonumber
\end{eqnarray}
%
where we have used the initial condition that $b=0$ at $t=0$. Then, using the results in section \ref{sec_lemma}, we have:
%
\begin{eqnarray}
\label{equ_ref}
\frac{b}{Q}  & = & \frac{\sigma}{2} \sqrt{ \frac{\pi}{2} } \sum_{m_z = 1}^{\infty} b_{m_z} \frac{1}{ c }\sin \left( \frac{m_z \pi} {H} z\right) \left(  \erf \left( \frac{x}{\sqrt{2} \sigma} \right) +  \erf \left( \frac{ct-x}{\sqrt{2} \sigma} \right)    \right)  \\ \nonumber
& + & \frac{\sigma}{2}  \sqrt{ \frac{\pi}{2} }  \sum_{m_z = 1}^{\infty} b_{m_z} \frac{ 1 }{ c } \sin \left( \frac{m_z \pi} {H} z\right)\left( - \erf \left( \frac{x}{\sqrt{2} \sigma} \right) + \erf \left( \frac{ct+x }{\sqrt{2} \sigma} \right)    \right) \\ \nonumber
& - & \frac{\sigma}{2}  \sqrt{ \frac{\pi}{2} }H(t-T) \sum_{m_z = 1}^{\infty} b_{m_z}  \frac{1 }{ c }  \sin \left( \frac{m_z \pi} {H} z\right) \left(- \erf \left( \frac{cT-\xi}{\sqrt{2} \sigma} \right) +  \erf \left( \frac{ct-\xi}{\sqrt{2} \sigma} \right)    \right)  \\ \nonumber
& - & \frac{\sigma }{2} \sqrt{ \frac{\pi}{2} }H(t-T)  \sum_{m_z = 1}^{\infty} b_{m_z}  \frac{1}{ c }  \sin \left( \frac{m_z \pi} {H} z\right) \left(- \erf \left( \frac{cT+\xi'}{\sqrt{2} \sigma} \right) +  \erf \left( \frac{ct +\xi'}{\sqrt{2} \sigma} \right)    \right), \\ \nonumber
\end{eqnarray}
%
where, e.g. in the second line on the right hand side we have used property \ref{equ_property1}. 
%
%
%
\subsubsection{Matlab Code}
%
Now, the Matlab code uses the above result as follows. Substitute for $\xi$ and $\xi'$:
%
\begin{eqnarray}
\frac{b}{Q}  & = & \frac{\sigma}{2} \sqrt{ \frac{\pi}{2} } \sum_{m_z = 1}^{\infty} b_{m_z} \frac{1}{ c }\sin \left( \frac{m_z \pi} {H} z\right) \left(  \erf \left( \frac{x}{\sqrt{2} \sigma} \right) +  \erf \left( \frac{ct-x}{\sqrt{2} \sigma} \right)    \right)  \\ \nonumber
& + & \frac{\sigma}{2}  \sqrt{ \frac{\pi}{2} }  \sum_{m_z = 1}^{\infty} b_{m_z} \frac{ 1 }{ c } \sin \left( \frac{m_z \pi} {H} z\right)\left( - \erf \left( \frac{x}{\sqrt{2} \sigma} \right) + \erf \left( \frac{ct+x }{\sqrt{2} \sigma} \right)    \right) \\ \nonumber
& - & \frac{\sigma}{2}  \sqrt{ \frac{\pi}{2} }H(t-T) \sum_{m_z = 1}^{\infty} b_{m_z}  \frac{1 }{ c }  \sin \left( \frac{m_z \pi} {H} z\right) \left(- \erf \left( \frac{-x}{\sqrt{2} \sigma} \right) +  \erf \left( \frac{c(t-T)-x}{\sqrt{2} \sigma} \right)    \right)  \\ \nonumber
& - & \frac{\sigma }{2} \sqrt{ \frac{\pi}{2} }H(t-T)  \sum_{m_z = 1}^{\infty} b_{m_z}  \frac{1}{ c }  \sin \left( \frac{m_z \pi} {H} z\right) \left(- \erf \left( \frac{x}{\sqrt{2} \sigma} \right) +  \erf \left( \frac{c(t-T) + x}{\sqrt{2} \sigma} \right)   \right), \\ \nonumber
\end{eqnarray}
%
and use the symmetry properties of the error function:
%
\begin{eqnarray}
\frac{b}{Q}  & = & \frac{\sigma}{2} \sqrt{ \frac{\pi}{2} } \sum_{m_z = 1}^{\infty} b_{m_z} \frac{1}{ c }\sin \left( \frac{m_z \pi} {H} z\right) \left(  \erf \left( \frac{x}{\sqrt{2} \sigma} \right) +  \erf \left( \frac{ct-x}{\sqrt{2} \sigma} \right)    \right)  \\ \nonumber
& + & \frac{\sigma}{2}  \sqrt{ \frac{\pi}{2} }  \sum_{m_z = 1}^{\infty} b_{m_z} \frac{ 1 }{ c } \sin \left( \frac{m_z \pi} {H} z\right)\left( - \erf \left( \frac{x}{\sqrt{2} \sigma} \right) + \erf \left( \frac{ct+x }{\sqrt{2} \sigma} \right)    \right) \\ \nonumber
& - & \frac{\sigma}{2}  \sqrt{ \frac{\pi}{2} } \sum_{m_z = 1}^{\infty} b_{m_z}  \frac{1 }{ c }  \sin \left( \frac{m_z \pi} {H} z\right) H(t-T)\left( \erf \left( \frac{x}{\sqrt{2} \sigma} \right) +  \erf \left( \frac{c(t-T)-x}{\sqrt{2} \sigma} \right)    \right)  \\ \nonumber
& - & \frac{\sigma }{2} \sqrt{ \frac{\pi}{2} }  \sum_{m_z = 1}^{\infty} b_{m_z}  \frac{1}{ c }  \sin \left( \frac{m_z \pi} {H} z\right) H(t-T)\left(- \erf \left( \frac{x}{\sqrt{2} \sigma} \right) +  \erf \left( \frac{c(t-T) + x}{\sqrt{2} \sigma} \right)    \right), \\ \nonumber
\end{eqnarray}
%
and collect terms, adding another factor of $\sigma$ to conserve total heating input (see below):
%
\begin{equation}
\frac{b}{Q}  = \frac{\sigma^2}{2} \sqrt{ \frac{\pi}{2} } \sum_{m_z = 1}^{\infty} b_{m_z} \frac{1}{ c }\sin \left( \frac{m_z \pi} {H} z\right) \left( F_1 (c) - F_1 (-c) \right) 
\end{equation}
%
where:
%
\begin{eqnarray}
F_1(c)  & \equiv & \erf \left( \frac{x}{\sqrt{2} \sigma} \right) +  \erf \left( \frac{ct-x}{\sqrt{2} \sigma} \right) \\ \nonumber
& - &  H(t-T) \left( \erf \left( \frac{x}{\sqrt{2} \sigma} \right) +  \erf \left( \frac{c(t-T)-x}{\sqrt{2} \sigma} \right)    \right) \\ \nonumber
\end{eqnarray}
%
%
%
\subsubsection{Compact Expression for $b$-Response}
%
To obtain a compact expression for $b$ let us return to equation \ref{equ_ref}. Cancelling terms and substituting for $\xi$ and $\xi'$:
%
\begin{eqnarray}
 \frac{2}{\sigma} \sqrt{ \frac{2} {\pi}} \frac{b}{Q}  & = &\sum_{m_z = 1}^{\infty} b_{m_z} \frac{1}{c} \sin \left( \frac{m_z \pi} {H} z\right) \erf \left( \frac{ct-x}{\sqrt{2} \sigma} \right)   \\ \nonumber
& + &\sum_{m_z = 1}^{\infty} b_{m_z} \frac{1}{c}  \sin \left( \frac{m_z \pi} {H} z\right) \erf \left( \frac{ct+x }{\sqrt{2} \sigma} \right) \\ \nonumber
& - & H(t-T) \sum_{m_z = 1}^{\infty} b_{m_z} \frac{1}{c} \sin \left( \frac{m_z \pi} {H} z\right) \left( - \erf \left( \frac{cT- x - cT}{\sqrt{2} \sigma} \right) +  \erf \left( \frac{ct-x-cT}{\sqrt{2} \sigma} \right)    \right)  \\ \nonumber
& - & H(t-T)  \sum_{m_z = 1}^{\infty} b_{m_z}  \frac{1}{c} \sin \left( \frac{m_z \pi} {H} z\right) \left( - \erf \left( \frac{cT+x-cT}{\sqrt{2} \sigma} \right) +  \erf \left( \frac{ct +x-cT}{\sqrt{2} \sigma} \right)    \right), \\ \nonumber
\end{eqnarray}
%
it is possible to simplify the above further I THINK (CHECKING). Recalling that the error function is an odd function:
%
\begin{eqnarray}
 \frac{2}{\sigma} \sqrt{ \frac{2} {\pi}} \frac{b}{Q}  & = &\sum_{m_z = 1}^{\infty}\frac{ b_{m_z}}{c}\sin \left( \frac{m_z \pi} {H} z\right) \erf \left( \frac{ct-x}{\sqrt{2} \sigma} \right)   \\ \nonumber
& + &\sum_{m_z = 1}^{\infty} \frac{ b_{m_z}}{c} \sin \left( \frac{m_z \pi} {H} z\right) \erf \left( \frac{ct+x }{\sqrt{2} \sigma} \right) \\ \nonumber
& - & H(t-T) \sum_{m_z = 1}^{\infty} \frac{ b_{m_z}}{c} \sin \left( \frac{m_z \pi} {H} z\right) \left( \erf \left( \frac{ x }{\sqrt{2} \sigma} \right) +  \erf \left( \frac{ct-x-cT}{\sqrt{2} \sigma} \right)    \right)  \\ \nonumber
& - & H(t-T)  \sum_{m_z = 1}^{\infty} \frac{ b_{m_z}}{c} \sin \left( \frac{m_z \pi} {H} z\right) \left( - \erf \left( \frac{x}{\sqrt{2} \sigma} \right) +  \erf \left( \frac{ct +x-cT}{\sqrt{2} \sigma} \right)    \right), \\ \nonumber
\end{eqnarray}
%
whereupon we can gain another cancellation:
%
\begin{eqnarray}
 \frac{2}{\sigma} \sqrt{ \frac{2} {\pi}} \frac{b}{Q}  & = &\sum_{m_z = 1}^{\infty}\frac{ b_{m_z}}{c} \sin \left( \frac{m_z \pi} {H} z\right) \erf \left( \frac{ct-x}{\sqrt{2} \sigma} \right)   \\ \nonumber
& + &\sum_{m_z = 1}^{\infty} \frac{ b_{m_z}}{c}  \sin \left( \frac{m_z \pi} {H} z\right) \erf \left( \frac{ct+x }{\sqrt{2} \sigma} \right) \\ \nonumber
& - & H(t-T) \sum_{m_z = 1}^{\infty}\frac{ b_{m_z}}{c} \sin \left( \frac{m_z \pi} {H} z\right) \erf \left( \frac{ct-x-cT}{\sqrt{2} \sigma} \right)  \\ \nonumber
& - & H(t-T)  \sum_{m_z = 1}^{\infty} \frac{ b_{m_z}}{c}  \sin \left( \frac{m_z \pi} {H} z\right) \erf \left( \frac{ct +x-cT}{\sqrt{2} \sigma} \right). \\ \nonumber
\end{eqnarray}
%
Finally, for the simplified potential temperature response, we have:
%
\begin{eqnarray}
 \frac{b}{Q}  & = & \frac{\sigma}{2} \sqrt{ \frac {\pi} {2}}  \sum_{m_z = 1}^{\infty} \frac{ b_{m_z}}{c} \sin \left( \frac{m_z \pi} {H} z\right) G(c,x,t)    \\ \nonumber
 & - & H(t-T) \frac{\sigma }{2} \sqrt{ \frac {\pi} {2}}  \sum_{m_z = 1}^{\infty}\frac{ b_{m_z}}{c} \sin \left( \frac{m_z \pi} {H} z\right) G(c,x,(t-T)), \\ \nonumber
 \end{eqnarray}
%
where:
%
\begin{equation}
 G(c,x,t)  \equiv \erf \left( \frac{ct-x}{\sqrt{2} \sigma} \right) +  \erf \left( \frac{ct+x }{\sqrt{2} \sigma} \right) .
\end{equation}
%
Note that to conserve the heating rate when $\sigma$ is varied:
%
\begin{equation}
F(x) \rightarrow \sigma \exp \left( - \frac{x^2}{2 \sigma^2 } \right), 
\end{equation}
%
whereupon the $b$-response acquires an additional factor $\sigma$:
%
\begin{eqnarray}
 \frac{b}{Q}  & = & \frac{\sigma^2}{2} \sqrt{ \frac {\pi} {2}}  \sum_{m_z = 1}^{\infty} \frac{ b_{m_z}}{c} \sin \left( \frac{m_z \pi} {H} z\right) G(c,x,t)    \\ \nonumber
 & - & H(t-T) \frac{\sigma^2}{2} \sqrt{ \frac {\pi} {2}} \sum_{m_z = 1}^{\infty}\frac{ b_{m_z}}{c} \sin \left( \frac{m_z \pi} {H} z\right) G(c,x,(t-T)), \\ \nonumber
 \end{eqnarray}
%


\end{document}